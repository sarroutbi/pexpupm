\documentclass[11pt]{article}
%Gummi|061|=)
\usepackage{hyperref}
\usepackage[spanish]{babel}
\usepackage[utf8]{inputenc}
\title{\textbf{Diseño de Software: Klondike}}
\author{Sergio Arroutbi Braojos}
\selectlanguage{spanish}
\date{\today}
%\addtolength{\topmargin}{-0.5in}
\usepackage[bottom=14em]{geometry}
\usepackage{amsmath}
\usepackage{mathtools}

\begin{document}

\hypersetup
{   
pdfborder={0 0 0}
}
   
\maketitle

\tableofcontents

\pagebreak

\section{Introducción}
En este ejercicio se ha realizado el análisis, diseño y posterior implementación del juego del klondike. Este juego de cartas, también conocido como ``solitario'', es un juego en el que un único jugador interviene en el reparto y operación de las cartas. El tablero de cartas está formado por la baraja inicial, conocida como ``Deck'', el ``Waste'', donde se levantan las cartas de la baraja, una serie de ``Tableaus'', que permiten sacar de forma provisional las cartas del ``Waste'' ordenándolas por número descendente y alternando los colores de las cartas, y finalmente las ``Foundations'', que son cada uno de los palos del ``Deck'' en orden ascendente.
En cada turno, el jugador puede sacar cartas del ``Deck'' al ``Waste'', colocar cartas del ``Waste'' a los ``Tableaus'' o a las ``Foundations'' y sacar igualmente carta de las ``Foundations'' para seguir combinando cartas en los ``Tableaus''.
El motivo de este documento no es describir de forma muy detallada el juego, que puede consultarse en el siguiente enlace: \url{https://en.wikipedia.org/wiki/Klondike_(solitaire)}
Además, existen ciertas reglas que son configurables en el juego, como el número de cartas del ``Deck'' que se ocultan en cada ronda de la baraja al ``Waste'', el número de veces que el ``Waste'' se puede dar la vuelta hacia el ``Deck''.
Para realizar la implementación software de este juego de cartas se han realizado las diversas fases involucradas en el proceso de desarrollo unificado con diseño orientado a objetos. En este proceso se ha llevado a cabo la idea ``BDUF'' (Big Design Up Front), para el cual se han seguido las siguientes fases:

\begin{itemize}\itemsep0pt
\item{Captura de Requisitos}
\item{Análisis}
\item{Diseño}
\item{Implementación}
\item{Pruebas}
\end{itemize}

En los siguientes capítulos se describen cada una de las anteriores fases y la aproximación que se ha llevado a cabo con el objetivo de llegar a una implementación con un objetivo final: Llegar a un software con un buen diseño orientado a objetos, cuya mantenibilidad sea lo más sencilla posible, y caracterizado por las cuatro premisas que este tipo de diseño trata de alcanzar:

\begin{itemize}\itemsep0pt
\item{Abstracción}
\item{Encapsulación}
\item{Modularidad}
\item{Jerarquía}
\end{itemize}

Para ello, se ha procurado implentar una jerarquía de clases de la mayor calidad posible, caracterizadas por ser :
\begin{itemize}\itemsep0pt
\item{de Bajo Acoplamiento}. Es decir, tener poca relación de fuerza entre unas claes y otras.
\item{Cohesivas}. Por cohesión, nos referimos a la fuerza de ligadura entre los miembros de una clase.
\item{Suficientes}. Por suficiencia se entiende que la clase proporciona suficientes características para proporcionar lo que la abstracción supone cuando interactúe con ellas.
\item{Completas}. Las clases deben ser completas, de forma que capture todas las características que la abstracción supone de ellas.
\item{Primitivas}. De forma que las clases pueden ser implementadas de forma efectiva simplemente proporcionando acceso a las capas inferiores.
\end{itemize}

Con las anteriores características en mente se describe cada una de las distintas fases desarrolladas hasta la implementación final.

\pagebreak

\section{Captura de Requisitos}

Para la captura de requisitos no se ha seguido ninguna metodología especial. El hecho de que el juego Klondike esté ya documentado e implementado permite dar una idea del funcionamiento del mismo.

Por esto, la principal fuente de requisitos para el desarrollo del juego han sido, en este caso, la documentación existente:
Por otro lado, a modo de ``Hands-On'', se ha utilizado el juego ``aisleriot'', implementación del klondike en la distribución Ubuntu de Linux.
A parte de lo descrito anteriormente, se han presentado ciertas características adicionales que pueden suponer ``motivos de cambio'' a considerar en el diseño de la aplicación, como son:

\begin{enumerate}\itemsep0pt
\item{El juego debería estar basado en opciones de menú a través de texto}. Sin embargo, no debería resultar muy difícil el portado del juego para que tuviera interfaz gráfico.
\item{Se podrá cambiar de tipo de baraja (baraja de póquer/baraja española)}. En el caso en el que se utilice la baraja española, la aceptación de cartas en los ``Tableaus'' será por ``palo distinto'', en lugar de ``color distinto'', como ocurre con la baraja francesa.
\item{El diseño debería facilitar el poder cambiar el tipo de reparto de cartas en el Deal de forma sencilla}.
\item{El diseño debería facilitar que el número de Tableaus se pudiera modificar de forma sencilla}.
\end{enumerate}


\end{document}
